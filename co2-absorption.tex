% Created 2024-05-21 Tue 11:43
\documentclass[10pt,a4paper,titlepage]{article}
\usepackage[utf8]{inputenc}
\usepackage[T1]{fontenc}
\usepackage{fixltx2e}
\usepackage{graphicx}
\usepackage{booktabs}
\usepackage{longtable}
\usepackage{float}
\usepackage{wrapfig}
\usepackage{rotating}
\usepackage[normalem]{ulem}
\usepackage{amsmath}
\usepackage{textcomp}
\usepackage{marvosym}
\usepackage{wasysym}
\usepackage{amssymb}
\usepackage{hyperref}
\usepackage{listings}
\tolerance=1000
\def\today{\number\day\space\ifcase\month\or January\or February\or March\or April\or May\or June\or July\or August\or September\or October\or November\or December\fi \space \number\year}
\usepackage{lmodern}
\usepackage{amssymb,amsmath}
\usepackage{parskip}
\usepackage[margin=1in]{geometry}
\usepackage[round]{natbib}
\usepackage{fancyhdr}
\usepackage{titling}
\usepackage[squaren,cdot]{SIunits}
\usepackage{booktabs}
\pagestyle{fancy}
\renewcommand{\footrulewidth}{0.4pt}
\newcommand{\isotope}[2]{\ensuremath{{^{#1}\mathrm{#2}}}}
\lhead{}
\rhead{}
\lfoot{CO2 Absorption}
\cfoot{\thepage}
\rfoot{\thedate}
\author{Stefan Revets}
\pretitle{\flushleft\LARGE\bfseries\vskip 80mm}
\posttitle{\par}
\preauthor{\flushleft}
\postauthor{\par}
\predate{\flushleft}
\postdate{\par{For discussion}}
\author{Stefan Revets}
\date{\today}
\title{CO$_{\text{2}}$ Absorption and Earth's Black Body Radiation}
\hypersetup{
  pdfkeywords={},
  pdfsubject={},
  pdfcreator={Emacs 24.5.1 (Org mode 8.2.10)}}
\begin{document}

\maketitle
\tableofcontents


\section{Introduction}
\label{sec-1}
The heating of the Earth due to human CO$_{\text{2}}$ emissions is the subject
of substantial research activities. A great deal of complex modelling
is carried out and used to inform humanity of various scenarios of
future climates. The Paris accord in 2015 accepted to bring down
CO$_{\text{2}}$ emissions so to limit the rise of the Earth's average
temperature to \unit{1.5}{\celsius}. The failure of politicians and
industrialists to take swift and sufficient action has led to the
prediction that the increase will be close to \unit{3}{\celsius}.

The modelling behind these numbers has overlooked a number of physical
facts and processes which have led to unrealistically optimistic
projections. A first-principles analysis uncovers a process of heating
and a feedback mechanism with very different consequences for the
future average Earth's temperature.

\section{Rationale}
\label{sec-2}
Earth receives essentially all its energy from the Sun (heat loss due
to cooling of the core and radiogenic heating amounts to
\unit{42}{\terad\watt}, compared to the \unit{174}{\petad\watt}
for the total solar energy delivered to the Earth). 

This received energy undergoes a number of conversions before
ultimately being radiated out into space. The various conversion
processes have their own response time, and they interact with each
other. This results in a dynamic equilibrium, maintaining the
temperature on Earth within relatively narrow bounds.

A clear distinction has to be made between the changes in CO$_{\text{2}}$ and
CH$_{\text{4}}$ concentrations and climate oscillations before
industrialisation and what is taking place in the last few decades,
now that a massive injection of CO$_{\text{2}}$ has taken place, geologically
speaking instantaneously. It is questionable if, Pre-Anthropocene,
greenhouse gases drove climate change: there are fascinating
indications that they were part of feedback mechanisms which kept the
Earth's climate within relatively narrow limits. It is not at all
clear if the geological concentrations and the temperatures/ice
volumes can be used as a direct, simple proxy for the effects of the
human-caused concentrations.

\section{A first principles approach}
\label{sec-3}

The greenhouse effect which keeps the Earth warmer than expected is
due to the absorption of infrared radiation radiated out by the
Earth. This radiation can be modelled effectively as black body
radiation. Therefore, I propose to analyse the heating up of the Earth
by simplifying the process to a black body surrounded by an
atmosphere, and singleing out the contribution of just CO$_{\text{2}}$, i.e.,
ignoring all the other component (including water vapour). It is,
after all, CO$_{\text{2}}$ that humans have been pumping in the atmosphere in
astounding amounts, and it is that what differentiates the present
situation from the remarkable stability of the Holocene.

I'll be using R to carry out most of the calculations, so let's bring
in the libraries needed and define a number of constants we'll be
using later on. The values of the various constants included have been
taken from CODATA 2010.

\lstset{literate={<-}{{$\gets$}}1 {<=}{{$\leq$}}1 {>=}{{$\geq$}}1 {!=}{{$\neq$}}1 {<>}{{$\neq$}}1 {\%>\%}{{$\Leftarrow$}}1 {\%<>\%}{{$\Leftrightarrow$}}1,frame=single,language=R,label= ,caption= ,numbers=none}
\begin{lstlisting}
library("dplyr")
library("ggplot2")
library("magrittr")
library("pracma")
library("tidyr")

R_gas <- 8.314471               # J / K mol, Gas constant
N_A <- 6.02214129e23            # / mol, Avogadro constant
k_boltzmann <- 1.3806488e-23    # J / K, Boltzmann constant (R / N_A)
h_planck <- 6.62606957e-34      # J s, Planck constant
c_light <- 299792458            # m / s, speed of light
sigma_stefan <- 5.670373e-8     # W / m^2 K^4, Stefan-Boltzmann constant
p0 <- 101325                    # Pa, standard pressure
\end{lstlisting}

\subsection{Earth's Atmosphere}
\label{sec-3-1}
\citet{trenberth-smith05:atmosphere} provide a carefully argued
calculation of the total mass of the atmosphere.

The total mean mass amounts to \unit{5.1480}{\exad\kilogram} with an
annual range due to water vapour of \unit{1.5}{\petad\kilogram}. The
mean mass of water vapour is estimated as \unit{12.7}{\petad\kilogram}
and the dry air mass as \unit{5.1352}{\exad\kilogram}.

Composition of the atmosphere
\begin{center}
\begin{tabular}{lrr}
\toprule
component & fraction & mol mass\\
\midrule
N$_{\text{2}}$ & 0.7808 & 28.0134\\
O$_{\text{2}}$ & 0.2094 & 31.8888\\
Ar & 0.0093 & 39.948\\
CO$_{\text{2}}$ & 0.0004 & 43.8998\\
\bottomrule
\end{tabular}
\end{center}
which yields an average mol mass for air of 28.9395

Assuming a homogeneous mix throughout the atmospheric column, these
numbers allow us to calculate the absolute amount of CO$_{\text{2}}$ in the
atmosphere, given concentration measurements (with M mass, MM molar
mass and [x] concentration), i.e.,

\begin{equation}
\mathrm{M_{CO_2}} = \frac{\mathrm{M_{air}}}{\mathrm{MM_{air}}} [x_{\mathrm{CO_2}}] \mathrm{MM_{CO_2}}
\end{equation}

And so we get for the conditions of relevant geological times

\begin{center}
\begin{tabular}{lrr}
\toprule
 & CO$_{\text{2}}$ (ppm) & CO$_{\text{2}}$ (\petad\kilogram)\\
\midrule
Glacial & 180 & 1.4056\\
Interglacial & 280 & 2.1866\\
Pliocene(?) & 350 & 2.7332\\
Present Day & 420 & 3.2799\\
\bottomrule
\end{tabular}
\end{center}

which suggests that the Glacial-Interglacial alternations correspond
to some \unit{781}{\terad\kilogram} CO$_{\text{2}}$ (or
\unit{17.79}{\petad\mole}) entering and leaving the atmosphere. It
also tells us that we have currently \unit{1.0933}{\petad\kilogram} in
excess of Interglacial amounts of CO$_{\text{2}}$ in the atmosphere.

The study by \citet{mcclatchey-al72:atmosphere} resulted in a set of
atmospheric profiles. They proposed a set of profiles by including
pressure, temperature, water and ozone content in function of height,
and that for tropical, mid-latitude and subarctic latitudes as well as
summer/winter differentiation. These profiles continue to be used,
e.g., \citet{mlynczak-al16:spectroscopic}.

Let's have a look at the data, plotting pressure (Pa),
temperature (K), and water content (\gram\per\metre\cubed).
\lstset{literate={<-}{{$\gets$}}1 {<=}{{$\leq$}}1 {>=}{{$\geq$}}1 {!=}{{$\neq$}}1 {<>}{{$\neq$}}1 {\%>\%}{{$\Leftarrow$}}1 {\%<>\%}{{$\Leftrightarrow$}}1,frame=single,language=R,label= ,caption= ,numbers=none}
\begin{lstlisting}
atm_equ <- read.table("data/grl54358_atmosphere_tropical.dat",
		       col.names = c("height", "pressure",
				     "temperature", "water", "ozone"),
		       skip = 2) %>%
    mutate(pressure = pressure * 100,
	   latitude = "Equatorial") %>%
    select(-ozone)

atm_mid_s <- read.table("data/grl54358_atmosphere_midlat_summer.dat",
			 col.names = c("height", "pressure",
				     "temperature", "water", "ozone"),
			 skip = 2) %>%
    mutate(pressure = pressure * 100,
	   latitude = "Temperate",
	   season = "Summer") %>%
    select(-ozone)

atm_mid_w <- read.table("data/grl54358_atmosphere_midlat_winter.dat",
			 col.names = c("height", "pressure",
				     "temperature", "water", "ozone"),
			 skip = 2) %>%
    mutate(pressure = pressure * 100,
	   latitude = "Temperate",
	   season = "Winter") %>%
    select(-ozone)

atm_pol_s <- read.table("data/grl54358_atmosphere_subarctic_summer.dat",
			 col.names = c("height", "pressure",
				     "temperature", "water", "ozone"),
			 skip = 2) %>%
    mutate(pressure = pressure * 100,
	   latitude = "Polar",
	   season = "Summer") %>%
    select(-ozone)

atm_pol_w <- read.table("data/grl54358_atmosphere_subarctic_winter.dat",
			 col.names = c("height", "pressure",
				     "temperature", "water", "ozone"),
			 skip = 2) %>%
    mutate(pressure = pressure * 100,
	   latitude = "Polar",
	   season = "Winter") %>%
    select(-ozone)

atmosphere <- atm_equ %>%
    gather("property", "value", 2:4) %>%
    bind_rows(gather(atm_mid_s, "property", "value", 2:4)) %>%
    bind_rows(gather(atm_mid_w, "property", "value", 2:4)) %>%
    bind_rows(gather(atm_pol_s, "property", "value", 2:4)) %>%
    bind_rows(gather(atm_pol_w, "property", "value", 2:4))

ggplot(atmosphere) +
    geom_line(aes(height, value, colour = season)) +
    facet_wrap(~latitude+property, scales = "free_y") +
    theme(legend.position = "bottom") +
    labs(x = "Height (km)")
\end{lstlisting}

\includegraphics[width=.9\linewidth]{output/atmosphere_profile.pdf}

The data determining these profiles will be very useful when
calculating the infrad red absorption of CO$_{\text{2}}$ in the atmospheric
column.

\subsection{The Sun as a Black Body}
\label{sec-3-2}
The Sun radiates energy out, the spectrum of which can be described
remarkably well as that of a black body. That principle can of course
also be applied to the Earth.

Black body radiation was first analysed and described by
\citet{planck00:energieverteilung} as
\begin{equation}
u(\nu, T) = \frac{8 \pi \nu^2}{c^3} \frac{h \nu}{e^{h \nu / k T} - 1}
\end{equation}
with $\nu$ the frequency of the radiation. Integrating over all the
frequencies yields
\begin{equation}
\begin{split}
U(T) & = \int_0^\infty \frac{8 \pi \nu^2}{c^3} \frac{h \nu}{e^{h \nu / k T} - 1} d \nu\\
     & = \frac{8 \pi^2}{60 c^3 h^3}k^4 T^4 \\
     & = \frac{4}{c} \sigma T^4
\end{split}
\end{equation}
with $\sigma$ the Stefan-Boltzman constant (as a combination of the
other constants, we can calculate this very easily as
\unit{56.70373}{\nano\watt\per\metre\squared\fourth\kelvin}). Note
that the usual relation for black body radiation
\begin{equation}
R(T) = \sigma T^4
\end{equation}
is the result of additional, and very much needed, integration over
all the angles of radiation of the previous integration: that one
addresses the ``uni-directional'' flux of energy.

There is a caveat to be heeded here. The relation between frequency
and wavelength cannot be used as is to convert the calculations from
frequency-based values to wavelength-based ones, that is to say
\begin{equation}
\begin{split}
u(\lambda, T) d\lambda & = -u(\epsilon, T) d\epsilon \\
\frac{d\epsilon}{d\lambda} & = -\frac{h c}{\lambda^2}
\end{split}
\end{equation}
and hence we get
\begin{equation}
u(\lambda, T) = \frac{8 \pi h c}{\lambda^5} \frac{1}{e^{h c /\lambda k T} - 1}
\end{equation}

Let's see this distribution for the Sun, with a temperature of some
\unit{5778}{\kelvin} 
\lstset{literate={<-}{{$\gets$}}1 {<=}{{$\leq$}}1 {>=}{{$\geq$}}1 {!=}{{$\neq$}}1 {<>}{{$\neq$}}1 {\%>\%}{{$\Leftarrow$}}1 {\%<>\%}{{$\Leftrightarrow$}}1,frame=single,language=R,label= ,caption= ,numbers=none}
\begin{lstlisting}
T_sun <- 5778
a1 <- 8 * pi * h_planck * c_light
a2 <- h_planck * c_light / (k_boltzmann * T_sun)

sun_energy <- data.frame(lambda = seq(100, 5000, 10)*1e-9) %>%
    mutate(Energy = a1 / (lambda^5 * (exp(a2 / lambda) - 1)))

ggplot(sun_energy) +
    geom_line(aes(lambda*1e9, Energy)) +
    labs(title = "Solar Radiation, Black Body at 5778 K",
	 x = "Wavelength (nm)",
	 y = "Energy (J/m3.m)")
\end{lstlisting}

\includegraphics[width=.9\linewidth]{output/solar_spectrum.pdf}

Let's check these calculations and equations, and compare the
(numerical) integration of the curve with the calculated energy, i.e.,
\lstset{literate={<-}{{$\gets$}}1 {<=}{{$\leq$}}1 {>=}{{$\geq$}}1 {!=}{{$\neq$}}1 {<>}{{$\neq$}}1 {\%>\%}{{$\Leftarrow$}}1 {\%<>\%}{{$\Leftrightarrow$}}1,frame=single,language=R,label= ,caption= ,numbers=none}
\begin{lstlisting}
sun_integral <- (sum(sun_energy$Energy) * 1e-8 )
sun_blackbody <- sigma_stefan * T_sun^4

(sun_integral * c_light / 4) / sun_blackbody
\end{lstlisting}

\begin{verbatim}
[1] 0.99478
\end{verbatim}

That confirms the numerical calculations: in addition, a dimensional
analysis likewise confirms the correctness of the need for the speed
of light as part of the final calculation
(\joule\per\metre\cubed $\times$ \metre\per\second $=$
\watt\per\metre\squared).

In order to calculate the actual energy distribution arriving on
Earth, we need to scale the black body spectrum just calculated, that
is to say, reduce it to the amount of energy arriving at the
Earth. 
And so we can confirm that that we do have
\unit{341.57}{\watt\per\metre\squared} coming in, averaged over the
globe and over the year. That allows us now to scale the energy of the
solar spectrum arriving at the Earth
\lstset{literate={<-}{{$\gets$}}1 {<=}{{$\leq$}}1 {>=}{{$\geq$}}1 {!=}{{$\neq$}}1 {<>}{{$\neq$}}1 {\%>\%}{{$\Leftarrow$}}1 {\%<>\%}{{$\Leftrightarrow$}}1,frame=single,language=R,label= ,caption= ,numbers=none}
\begin{lstlisting}
solar_incoming <- 341.57
scale_factor <- solar_incoming * c_light / (4 * sigma_stefan * T_sun^4)

sun_energy %<>%
    mutate(Energy = Energy * scale_factor)

ggplot(sun_energy) +
    geom_line(aes(lambda*1e9, Energy)) +
    labs(title = "Incoming Solar Radiation profile",
	 x = "Wavelength (nm)",
	 y = "Energy (W/m3)")
\end{lstlisting}

\includegraphics[width=.9\linewidth]{output/solar_spectrum_scaled.pdf}

A quick check to see if this normalisation is correct:
\lstset{literate={<-}{{$\gets$}}1 {<=}{{$\leq$}}1 {>=}{{$\geq$}}1 {!=}{{$\neq$}}1 {<>}{{$\neq$}}1 {\%>\%}{{$\Leftarrow$}}1 {\%<>\%}{{$\Leftrightarrow$}}1,frame=single,language=R,label= ,caption= ,numbers=none}
\begin{lstlisting}
sum(sun_energy$Energy) * 1e-8
\end{lstlisting}

\begin{verbatim}
[1] 339.787
\end{verbatim}

which converged sufficiently close to the expected value of
\unit{341.57}{\watt\per\metre\squared}.

The amount of incoming solar radiation gives us also a means of
assessing to what extent the Earth can be considered as a black
body. As the amount of energy arriving at the Earth has to be radiated
out again, otherwise the Earth's temperature would continue to
increase, we can calculate the expected black body temperature as
\lstset{literate={<-}{{$\gets$}}1 {<=}{{$\leq$}}1 {>=}{{$\geq$}}1 {!=}{{$\neq$}}1 {<>}{{$\neq$}}1 {\%>\%}{{$\Leftarrow$}}1 {\%<>\%}{{$\Leftrightarrow$}}1,frame=single,language=R,label= ,caption= ,numbers=none}
\begin{lstlisting}
(solar_incoming / sigma_stefan)^(1/4)
\end{lstlisting}

\begin{verbatim}
[1] 278.591
\end{verbatim}

which is about \unit{10}{\kelvin} below the current average Earth
temperature of \unit{287}{\kelvin}. So, as expected, the Earth does
not behave as an ideal black body. Furthermore,
\citet{trenberth-al09:budget} inform us that about 30\% of the incoming
energy is reflected (the albedo, a combination of cloud, ocean, ice
and land reflectivity). Using this downward revised number out would
result in a, even lower black body temperature, i.e.,
\unit{254.8}{\kelvin}. Clearly, the greenhouse effect due to the
atmospheric water and CO$_{\text{2}}$ content plays a vital part in generating
the Earth's temperature.

At the same time, it is perfectly clear from the energy distribution
of the incoming solar radiation just plotted that very little of that
energy is available for the absorption bands of CO$_{\text{2}}$.

As an aside, we can calculate in principle the expected black body
temperature of any of the planets, as some elementary algebra tells us
that
\begin{equation}
T_{\mathrm{Planet}} = \sqrt{\frac{R_{\mathrm{Sun}}}{2 R_{\mathrm{orbit}}}} \, T_{\mathrm{Sun}}
\end{equation}
assuming of course that any planetary albedo will be zero

So, given a solar radius of \unit{695508}{\kilo\metre} and a
temperature of \unit{5778}{\kelvin}, we predict

\begin{center}
\begin{tabular}{lrr}
\toprule
Planet & distance (km) & T$_{\text{Black Body}}$ (K)\\
\midrule
Mercury & 57909227 & 447.75\\
Venus & 108209475 & 327.55\\
Earth & 149598262 & 278.58\\
Mars & 227943824 & 225.68\\
Jupiter & 778340821 & 122.13\\
Saturn & 1426666422 & 90.20\\
\bottomrule
\end{tabular}
\end{center}

\subsection{Earth as a Black Body}
\label{sec-3-3}
Turning to the Earth as a black body radiator, we can repeat the
earlier calculations for the solar energy, but this time for the Earth
(taking an average temperature of \unit{287}{\kelvin}), and using the
estimated value of \unit{238.5}{\watt\per\metre\squared} of outgoing
radiation \citep{trenberth-al09:budget}.
\lstset{literate={<-}{{$\gets$}}1 {<=}{{$\leq$}}1 {>=}{{$\geq$}}1 {!=}{{$\neq$}}1 {<>}{{$\neq$}}1 {\%>\%}{{$\Leftarrow$}}1 {\%<>\%}{{$\Leftrightarrow$}}1,frame=single,language=R,label= ,caption= ,numbers=none}
\begin{lstlisting}
T_earth <- 287
earth_outgoing <- 238.5
scale_factor <- earth_outgoing * c_light / (4 * sigma_stefan * T_earth^4)

a1 <- 8 * pi * h_planck * c_light
a2 <- h_planck * c_light / (k_boltzmann * T_earth)

earth_energy <- data.frame(lambda = seq(2000, 50000, 200)*1e-9) %>%
    mutate(Energy = a1 / (lambda^5 * (exp(a2 / lambda) - 1)),
	   Energy = Energy * scale_factor)

ggplot(earth_energy) +
    geom_line(aes(lambda*1e6, Energy)) +
    labs(title = "Earth Radiation, Black Body at 288 K",
	 x = expression(paste("Wavelength (", mu, "m)")),
	 y = "Energy (W/m2)")
\end{lstlisting}

\includegraphics[width=.9\linewidth]{output/earth_spectrum.pdf}

and so we see that a great deal of (outgoing) energy can be absorbed
by CO$_{\text{2}}$, i.e., the actual greenhouse effect.

\subsection{The role of CO$_{\text{2}}$}
\label{sec-3-4}
\subsubsection{Energy Absorption}
\label{sec-3-4-1}
Let's now turn to the energy absorption by CO$_{\text{2}}$, and, to a lesser
extent, that of H$_{\text{2}}$O in so far as its absorption overlaps or
interferes with that of CO$_{\text{2}}$.

A great deal of meticulous effort has been expended on obtaining these
absorption spectra, including quantum mechanical calculations
\citep{rothman-al2009:hitran,lamouroux-al12:database,etminan-al16:forcing,mlynczak-al16:spectroscopic}. As
a result, we have now access to a great deal of precision,
outstripping the earlier, necessarily more crude broader absorption
peaks.  These efforts have culminated into a database which can be
accessed on-line (\url{https://hitran.org/}) to obtain any desired data.
The level of detail captured in the database includes the separation
of each isotope of the constituent atoms. The most abundant isotopic
mix is marked with code number 1.

A complete explanation of the  HITRAN data structure is as follows:

\begin{center}
\begin{tabular}{lrll}
\toprule
parameter & field length & format & units\\
\midrule
M & 2 & I2 & HITRAN molecule number code\\
I & 1 & I1 & ordering within molecule by terrestrial abundance\\
$\nu$ & 12 & F12.6 & \centi\reciprocal\metre\\
S & 10 & E10.3 & \centi\reciprocal\metre\per\centi\rpsquare\metre molecule\\
A & 10 & E10.3 & \reciprocal\second\\
$\gamma$$_{\text{air}}$ & 5 & F5.4 & HWHM at \unit{296}{\kelvin} (\centi\reciprocal\metre atm$^{\text{-1}}$)\\
$\gamma$$_{\text{self}}$ & 5 & F5.4 & HWHM at \unit{296}{\kelvin} (\centi\reciprocal\metre atm$^{\text{-1}}$)\\
E'' & 10 & F10.4 & \centi\reciprocal\metre\\
n$_{\text{air}}$ & 4 & F4.2 & exponent of $\gamma$$_{\text{air}}$\\
$\delta$$_{\text{air}}$ & 8 & F8.6 & \centi\reciprocal\metre atm$^{\text{-1}}$\\
V' & 15 & A15 & Upper state global quanta\\
V'' & 15 & A15 & Lower state global quanta\\
Q' & 15 & A15 & Upper state local quanta\\
Q'' & 15 & A15 & Lower state local quanta\\
Ierr & 6 & 6I1 & accuracy codes for $\nu$, S, $\gamma$$_{\text{air}}$, $\gamma$$_{\text{self}}$, n$_{\text{air}}$, $\delta$$_{\text{air}}$\\
Iref & 12 & 6I2 & references for $\nu$, S, $\gamma$$_{\text{air}}$, $\gamma$$_{\text{self}}$, n$_{\text{air}}$, $\delta$$_{\text{air}}$\\
flag & 1 & A1 & \\
g' & 7 & F7.1 & weight of upper state\\
g'' & 7 & F7.1 & weight of lower state\\
\bottomrule
\end{tabular}
\end{center}

The two molecules on interest in the analysis here are H$_{\text{2}}$O and
CO$_{\text{2}}$. A query of the database for the most common isotopologues of
these two molecules and including all the wavelengths for which data
is available, yielded 319887 entries for H$_{\text{2}}$O and 174446 entries for
CO$_{\text{2}}$.

A neat way of bringing in the data is through an AWK script. Although
the data is presented in a Fortran-type format, i.e., columnar, there
is a possibility of defining fields by their width, rather than the
usual AWK field separator character.

First, to get the relevant data for CO$_{\text{2}}$
\lstset{literate={<-}{{$\gets$}}1 {<=}{{$\leq$}}1 {>=}{{$\geq$}}1 {!=}{{$\neq$}}1 {<>}{{$\neq$}}1 {\%>\%}{{$\Leftarrow$}}1 {\%<>\%}{{$\Leftrightarrow$}}1,frame=single,language=awk,label= ,caption= ,numbers=none}
\begin{lstlisting}
BEGIN{OFS = ","
    FIELDWIDTHS = "2 1 12 10 10 5 5 10 4 8 60 6 12 1 7 7"
      print "nu, intensity, Einstein_A, gamma_air, gamma_self, \
	     energy_lower, n_air, delta_air"}
    {print $3, $4, $5, $6, $7, $8, $9, $10
}
\end{lstlisting}

and similarly, for H$_{\text{2}}$O
\lstset{literate={<-}{{$\gets$}}1 {<=}{{$\leq$}}1 {>=}{{$\geq$}}1 {!=}{{$\neq$}}1 {<>}{{$\neq$}}1 {\%>\%}{{$\Leftarrow$}}1 {\%<>\%}{{$\Leftrightarrow$}}1,frame=single,language=awk,label= ,caption= ,numbers=none}
\begin{lstlisting}
BEGIN{OFS = ","
    FIELDWIDTHS = "2 1 12 10 10 5 5 10 4 8 60 6 12 1 7 7"
      print "nu, intensity, Einstein_A, gamma_air, gamma_self, \
	     energy_lower, n_air, delta_air"}
    {print $3, $4, $5, $6, $7, $8, $9, $10
}
\end{lstlisting}

Let's have a look at the absorption data, and in particular, compare
and contrast the H$_{\text{2}}$O and CO$_{\text{2}}$ absorption spectra.

\lstset{literate={<-}{{$\gets$}}1 {<=}{{$\leq$}}1 {>=}{{$\geq$}}1 {!=}{{$\neq$}}1 {<>}{{$\neq$}}1 {\%>\%}{{$\Leftarrow$}}1 {\%<>\%}{{$\Leftrightarrow$}}1,frame=single,language=R,label= ,caption= ,numbers=none}
\begin{lstlisting}
hitran_h2o <- read.csv("cache/hitran_h2o.dat")
hitran_co2 <- read.csv("cache/hitran_co2.dat")

hitran_abs <- hitran_h2o %>%
    mutate(molecule = "H2O") %>%
    bind_rows(hitran_co2 %>%
	      mutate(molecule = "CO2")) %>%
    select(nu, intensity, molecule)

hitran_abs %>%
    filter(intensity > 1e-24) %>%
    ggplot() +
    geom_line(aes(nu, intensity)) +
    facet_wrap(~molecule, nrow=2) +
    xlim(0, 10000) +
    scale_y_log10() +
    labs(title = "HITRAN Absorption Intensities",
	 x = expression(paste("Wave number ", (cm^{-1}))),
	 y = "Log Intensity")
\end{lstlisting}

\includegraphics[width=.9\linewidth]{output/absorption.pdf}

This plot shows rather nicely that there are essentially three regions
in which CO$_{\text{2}}$ absorption can take place more or less unhindered by
the presence of water vapour (around 667, 2360 and
\unit{5000}{\centi\reciprocal\metre}; the bands around 3600 and
\unit{3700}{\centi\reciprocal\metre} appear to be largely saturated by
the H$_{\text{2}}$O absorption).

Taking a closer look at these four regions
\lstset{literate={<-}{{$\gets$}}1 {<=}{{$\leq$}}1 {>=}{{$\geq$}}1 {!=}{{$\neq$}}1 {<>}{{$\neq$}}1 {\%>\%}{{$\Leftarrow$}}1 {\%<>\%}{{$\Leftrightarrow$}}1,frame=single,language=R,label= ,caption= ,numbers=none}
\begin{lstlisting}
hitran_abs %>%
    filter(nu > 590 & nu < 740) %>%
    mutate(region = 665) %>%
    bind_rows(filter(hitran_abs, nu > 2265 & nu < 2415) %>%
	      mutate(region = 2340)) %>%
    bind_rows(filter(hitran_abs, nu > 3550 & nu < 3800) %>%
	      mutate(region = 3625)) %>%
    bind_rows(filter(hitran_abs, nu > 4800 & nu < 5050) %>%
	      mutate(region = 4900)) %>%
    filter(intensity > 1e-25) %>%
    ggplot() +
    geom_line(aes(nu, intensity, colour = molecule)) +
    facet_wrap(~region,scales="free") +
    labs(title = "HITRAN Absorption Intensities",
	 x = expression(paste("Wave number ", (cm^{-1}))),
	 y = "Intensity") +
    theme(legend.position = "bottom")
\end{lstlisting}

\includegraphics[width=.9\linewidth]{output/absorption_details.pdf}

we get a first inkling of the relative importance of the different
bands (note that the Y-axis is now linear, while in the previous plot
it was logarithmic). But we also need to bring into consideration the
amount of energy that can be absorbed, i.e., where these absorption
bands fall in the Earth's black body spectrum.

An first glimpse can be had by superposing the Earth's black body
radiation profile on the CO$_{\text{2}}$ absorption bands

\lstset{literate={<-}{{$\gets$}}1 {<=}{{$\leq$}}1 {>=}{{$\geq$}}1 {!=}{{$\neq$}}1 {<>}{{$\neq$}}1 {\%>\%}{{$\Leftarrow$}}1 {\%<>\%}{{$\Leftrightarrow$}}1,frame=single,language=R,label= ,caption= ,numbers=none}
\begin{lstlisting}
co2_spectrum <- hitran_co2 %>%
    mutate(lambda = 0.01/nu) %>%
    select(lambda, intensity)

ggplot(earth_energy) +
    geom_line(aes(lambda*1e6, intensity*1e25),
	      data = co2_spectrum) +
    geom_line(aes(lambda*1e6, Energy), color = "red") +
    scale_y_continuous(sec.axis =
			   sec_axis(~. /1e25, name = "CO2 Absorption")) +
    labs(title = "Earth's Black Body Radiation (288 K) and CO2 Absorption",
	 x = expression(paste("Wavelength (", mu, "m)")),
	 y = "Energy (W/m2)")
\end{lstlisting}

\includegraphics[width=.9\linewidth]{output/bb_radiation_co2.pdf}

Let's quantify the differences glimpsed on the plot and calculate the
energy available to each of these absorption bands
\lstset{literate={<-}{{$\gets$}}1 {<=}{{$\leq$}}1 {>=}{{$\geq$}}1 {!=}{{$\neq$}}1 {<>}{{$\neq$}}1 {\%>\%}{{$\Leftarrow$}}1 {\%<>\%}{{$\Leftrightarrow$}}1,frame=single,language=R,label= ,caption= ,numbers=none}
\begin{lstlisting}
earth_avail <- data.frame(lambda = seq(13.50,16.95,0.1)*1e-6,
			  band = "0665", stringsAsFactors = FALSE) %>%
    bind_rows(data.frame(lambda = seq(4.14,4.42,0.02)*1e-6,
			 band = "2340", stringsAsFactors = FALSE)) %>%
    bind_rows(data.frame(lambda = seq(2.63,2.82,0.01)*1e-6,
			 band = "3625", stringsAsFactors = FALSE)) %>%
    bind_rows(data.frame(lambda = seq(1.97,2.03,0.005)*1e-6,
			 band = "5000", stringsAsFactors = FALSE))

lambda_steps <- data.frame(band = c("0665", "2340", "3625", "5000"),
			   stepsize = c(0.1e-6, 0.02e-6, 0.01e-6, 0.005e-6),
			   stringsAsFactors = FALSE)

T_earth <- 287
earth_outgoing <- 238.5
scale_factor <- earth_outgoing * c_light / (4 * sigma_stefan * T_earth^4)
a1 <- 8 * pi * h_planck * c_light
a2 <- h_planck * c_light / (k_boltzmann * T_earth)

earth_avail %<>%
    mutate(energy = a1 / (lambda^5 * (exp(a2 / lambda) - 1)),
	   energy = energy * scale_factor) %>%
    group_by(band) %>%
    summarise_at("energy", sum) %>%
    ungroup() %>%
    left_join(lambda_steps, by = "band") %>%
    mutate(energy = energy * stepsize) %>%
    select(-stepsize)
\end{lstlisting}

\begin{center}
\begin{tabular}{lrr}
\toprule
wave number \centi\reciprocal\metre & \textmu{}\metre & energy (\watt\per\metre\squared)\\
\midrule
665 (590-740) & 16.95-13.51 & 38.478\\
2340 (2265-2415) & 4.42-4.14 & 0.399\\
3625 (3550-3800) & 2.82-2.63 & 0.0033\\
5000 (4925-5075) & 2.03-1.97 & 0.0000062\\
\bottomrule
\end{tabular}
\end{center}

which shows the significant differences between the different
regions. Clearly, the region of the
\unit{665}{\centi\reciprocal\metre} is almost a hundred times larger
than the next band, and even more so for other regions. It would be
wise, then, to investigate this region more closely.

On the subject of energy available for absorption, let's take the
opportunity to record the relation between black body temperature and
energy available for a given absorption band. Given different earth
surface temperatures, how much energy is then radiated out again in
the CO$_{\text{2}}$ absorption regions of interest? We can get the values from
simply running the code snippet just above for a suite of
temperatures: that gives for the \unit{665}{\centi\reciprocal\metre}
band

\begin{table}[htb]
\centering
\begin{tabular}{rr}
\toprule
T (\kelvin) & Energy (\watt\per\metre\squared)\\
\midrule
265 & 39.77\\
270 & 39.54\\
275 & 39.27\\
280 & 38.95\\
285 & 38.62\\
290 & 38.26\\
295 & 37.87\\
300 & 37.46\\
305 & 37.03\\
310 & 36.59\\
\bottomrule
\end{tabular}\caption{\label{energy665}Temperature dependence of energy available for absorption}

\end{table}

\lstset{literate={<-}{{$\gets$}}1 {<=}{{$\leq$}}1 {>=}{{$\geq$}}1 {!=}{{$\neq$}}1 {<>}{{$\neq$}}1 {\%>\%}{{$\Leftarrow$}}1 {\%<>\%}{{$\Leftrightarrow$}}1,frame=single,language=R,label= ,caption= ,numbers=none}
\begin{lstlisting}
names(energy665) <- c("T","Energy")

ggplot(energy665) +
    geom_line(aes(T, Energy)) +
    labs(title = "Energy available in 665 band",
	 x = "T (K)",
	 y = "Energy (W/m2)")
\end{lstlisting}

\includegraphics[width=.9\linewidth]{output/energy665.pdf}

The paradox of larger amounts of energy available at lower temperature
(within the range explored here) where one would expect the opposite
is simply due to the shape of the black body curve and the spot
occupied by the band of interest. The relevance of this
counterintuitive finding lies with the suggestion that more energy is
available for absorption by CO$_{\text{2}}$ in the colder regions.

The shape of an absorption peak is not as straightforward a matter as
one might expect. An absorption line is broadened around its
transition wavelength due to temperature, pressure, and presence of
other molecules \citep{tennyson-al14:IUPAC}. In the lower atmosphere,
the broadening of a line is largely driven by collisions with other
molecules, and can be modeled by a Lorentz profile:
\begin{equation}
L(\nu; \nu_{ij}, p, T) = \frac{1}{\pi} \frac{\gamma(p, T)}{\gamma(p, T)^2 + (\nu - \nu^*_{ij})^2}
\end{equation}
where
\begin{equation}
\begin{split}
\gamma(p, T) & = \Bigg( \frac{T_0}{T} \Bigg)^{n_{air}} \bigg( \gamma_{air}(p_0, T_0)(p - p_{self}) + \gamma_{self}(p_0, T_0) p_{self} \bigg) \\
\nu_{ij}^* & = \nu_{ij} + \delta p_0 p
\end{split}
\end{equation}
In addition, the thermal translational movement of the molecules also
causes a Doppler effect, which shows up as a frequency shift. The
resulting profile can be described by a Gaussian curve, 
\begin{equation}
G(\nu - \nu_0) = \sqrt{\frac{\ln 2}{\pi}} \frac{1}{\Gamma_D} e^{-\ln 2 (\frac{\nu - \nu_0}{\Gamma_D})^2}
\end{equation}
with $\Gamma$$_{\text{D}}$ the Doppler half-width
\begin{equation}
\Gamma_D = \sqrt{\frac{2 \ln 2 k T}{m c^2}} \nu_0
\end{equation}
To account for both effects, these two profiles can be convoluted into
a Voigt Profile
\begin{equation}
I(\nu) = \int^{\infty}_{-\infty} G(\nu') L(\nu - \nu') d\nu'
\end{equation}
which needs to be evaluated numerically \citep{thompson93:voigt}.

Fortunately, a Fortran package with associated data files
\citep{gordon-al22:hitran2020,lamouroux-al15:linemixing} is freely
available from \url{https://hitran.org/supplementary/} and allows for easy
experimentation.  The following variables can be changed before
carrying out the calculations

\begin{center}
\begin{tabular}{lll}
\toprule
variable & unit & description\\
\midrule
SgMinR & \centi\reciprocal\metre & Minimum wavenumber\\
SgMaxR & \centi\reciprocal\metre & Maximum wavenumber\\
dSg & \centi\reciprocal\metre & Wavenumber step size\\
sTotMax &  & Band intensity cut-off\\
Temp & K & Temperature\\
Ptot & atm & Total pressure\\
xCO2 &  & CO$_{\text{2}}$ mole fraction\\
xH2O &  & H$_{\text{2}}$O mole fraction\\
MixFull & logical & Full line mixing calcs\\
MixSDV & logical & Speed-dependent Voigt profile\\
\bottomrule
\end{tabular}
\end{center}

I have modified the Fortran code a little so that these variables are
read in from a separate file (lm\_input.in), rather than having to
change the parameters in the code and then recompiling the code. As a
result, it becomes much easier to carry out more numerous
calculations.

The results of the calculations all go into a file
(ABSCO\_HIT2020.dat), all in \centi\reciprocal\metre. Looking more
closely at the Fortran code reveals that the absorption coefficient is
standardised to the number of CO$_{\text{2}}$ molecules per \metre\cubed, i.e.,
the mole fraction $x_{CO_2}$, temperature $T$ and pressure $p$ are
used to calculate the number of molecules: the absorption is then
multiplied with this ``density''.

\begin{equation}
d_{\mathrm{CO}_2} = \frac{p N_A x_{\mathrm{CO}_2}}{R T}
\end{equation}

The output format is as follows

\begin{center}
\begin{tabular}{ll}
\toprule
entry & description\\
\midrule
nu & wave number\\
AbsV & Absorption coeffcient using Voigt line shape, no line mixing\\
AbsY & Absorption coefficient, first-order line mixing\\
AbsW & Absorption coefficient, full diagonalisation line mixing\\
\bottomrule
\end{tabular}
\end{center}

As an example, let's calculate the absorption centered around an
absorption peak at \unit{15}{\micro\metre} (or, in the cgs units
used by the code, \unit{150}{\centi\reciprocal\metre} around
\unit{665}{\centi\reciprocal\metre}) with the other parameter values
as listed in the table

\begin{center}
\begin{tabular}{lll}
\toprule
variable & value & unit\\
\midrule
dSg & 0.01 & \centi\reciprocal\metre\\
sTotMax & 0.1 10$^{\text{-28}}$ & \\
Temp & 290 & K\\
Ptot & 1.0 & atm\\
xCO2 & 420 10$^{\text{-6}}$ & \\
xH2O & 0 & \\
MixFull & .False. & logical\\
MixSDV & .True. & logical\\
\bottomrule
\end{tabular}
\end{center}

We'll use shell scripts to store the desired parameters and their
values into the expected lm\_input.in file and then run the modified
lm\_calc program. Let's move the output of the calculations to the
cache directory, where we can access it for further scrutiny

\lstset{literate={<-}{{$\gets$}}1 {<=}{{$\leq$}}1 {>=}{{$\geq$}}1 {!=}{{$\neq$}}1 {<>}{{$\neq$}}1 {\%>\%}{{$\Leftarrow$}}1 {\%<>\%}{{$\Leftrightarrow$}}1,frame=single,language=sh,label= ,caption= ,numbers=none}
\begin{lstlisting}
cd hitran2020
cat > lm_input.in <<EOF
SgMinR  590.0
SgMaxR  740.0
dSg     0.01
sTotMax 0.1e-28
Temp    290
Ptot    1.0
xCO2    420.e-6
xH2O    0
MixFull 0
MixSDV  1
EOF

./lm_calc
mv ABSCO_HIT2020.dat ../cache/hitran_lm_665.dat
cd ..
\end{lstlisting}

And here is a plot of the results
\lstset{literate={<-}{{$\gets$}}1 {<=}{{$\leq$}}1 {>=}{{$\geq$}}1 {!=}{{$\neq$}}1 {<>}{{$\neq$}}1 {\%>\%}{{$\Leftarrow$}}1 {\%<>\%}{{$\Leftrightarrow$}}1,frame=single,language=R,label= ,caption= ,numbers=none}
\begin{lstlisting}
hitran_665 <- read.table("cache/hitran_lm_665.dat",
			  col.names = c("nu","Voigt","First_Order","Full_Calc"))

ggplot(hitran_665) +
    geom_line(aes(nu, Voigt)) +
    labs(title = "Absorption Coefficients, with Voigt Profile",
	 x = expression(paste('Wave number ', (cm^{-1}))),
	 y = expression(paste('Absorption Coefficient ', (cm^{-1}/m^3))))
\end{lstlisting}

\includegraphics[width=.9\linewidth]{output/hitran_665.pdf}

It is a straightforward matter to calculate the area under the curves
(sum of all the values multiplied with the step size), and so we can
assess the contributions of the various absorption bands.

\begin{center}
\begin{tabular}{rrrr}
\toprule
wave number \centi\reciprocal\metre & \textmu{}\metre & IAbs\_V & IAbs\_f\\
\midrule
665 & 15.0 & 0.09656 & 0.09659\\
2340 & 4.27 & 1.07276 & 1.07349\\
3625 & 2.76 & 0.01197 & 0.01197\\
3725 & 2.68 & 0.01775 & 0.01776\\
4845 & 2.06 & 0.00009 & 0.00009\\
5000 & 2.00 & 0.00040 & 0.00040\\
\bottomrule
\end{tabular}
\end{center}

As the intervals span \unit{150}{\centi\reciprocal\metre}, the maximum
area under the curve would stand at
\unit{150}{\centi\reciprocal\metre}, and the numbers above should be
divided by this. Clearly, the amount of radiation absorption at these
concentrations is small.

 We do have to put these numbers into the context of the amount of
energy available for absorption, already derived above.  Taking both
tables into consideration, we see that the bands around 3625 and
\unit{3727}{\centi\reciprocal\metre} (as we have already had some
indication of) are not only largely subsumed in some of the H$_{\text{2}}$O
absorption bands, they also account for barely 2.5\% of the overall
absorption, while the bands around
\unit{5000}{\centi\reciprocal\metre} account for even less than
that. In contrast, the energy available for absorption in the band
around \unit{665}{\centi\reciprocal\metre} is two order of magnitude
larger than that available to \unit{2340}{\centi\reciprocal\metre}, so
even though the absorption of the \unit{2340}{\centi\reciprocal\metre}
band is 10 times as strong as that of the
\unit{665}{\centi\reciprocal\metre} band, we can safely neglect all
but the \unit{665}{\centi\reciprocal\metre} band for further
consideration.

\citep{tennyson-al14:IUPAC} showed that temperature, pressure, and
presence of other molecules have an effect on the shape of the
absorption curves, and is reflected in the relevant equations. What is
not obvious is to what degree the absorption changes with the
variation in temperature and pressure as encountered in the
atmosphere. Fortunately, we are now in a position to explore these
effects on the integrated absorption. Perusal of the Lorentz and
Gau\ss profile equations above hint that concentration differences in
CO$_{\text{2}}$ will have very little effect on the individual absorption,
particularly at the ppm concentrations we are dealing with here.

First, let's look at the dependence between absorbance and pressure
\lstset{literate={<-}{{$\gets$}}1 {<=}{{$\leq$}}1 {>=}{{$\geq$}}1 {!=}{{$\neq$}}1 {<>}{{$\neq$}}1 {\%>\%}{{$\Leftarrow$}}1 {\%<>\%}{{$\Leftrightarrow$}}1,frame=single,language=sh,label= ,caption= ,numbers=none}
\begin{lstlisting}
cd hitran2020
rm -f ../cache/lm_calc_p.dat

cat > lm_input.stub <<EOF
SgMinR  590.0
SgMaxR  740.0
dSg     0.01
sTotMax 0.1e-28
Temp    290
xCO2    420.e-6
xH2O    0
MixFull 0
MixSDV  1
Ptot    1.0
EOF

for p in 1.0 0.5 0.1 0.05 0.01; do
    head -9 lm_input.stub > lm_input.in
    cat >> lm_input.in <<EOF
Ptot    $p
EOF

./lm_calc

awk -v press=$p '{AbsV += $2;AbsL += $3}END{print press, AbsV/100, AbsL/100}' \
    ABSCO_HIT2020.dat >> ../cache/lm_calc_p.dat

done
cd ..
\end{lstlisting}

Let's plot the results of the integrated speed-dependent Voigt
absorption, relative to the base case
\lstset{literate={<-}{{$\gets$}}1 {<=}{{$\leq$}}1 {>=}{{$\geq$}}1 {!=}{{$\neq$}}1 {<>}{{$\neq$}}1 {\%>\%}{{$\Leftarrow$}}1 {\%<>\%}{{$\Leftrightarrow$}}1,frame=single,language=R,label= ,caption= ,numbers=none}
\begin{lstlisting}
Press_Abs <- read.table("cache/lm_calc_p.dat",
			col.names = c("pressure", "AbsV", "AbsL"))

Press_Abs %>%
    mutate(Voigt = AbsV / (pressure * Press_Abs$AbsV[1])) %>%
    ggplot() +
    geom_point(aes(pressure * p0/1000, Voigt)) +
    labs(title = "Relative Absorbance in function of Pressure",
	 x = "Pressure (kPa)",
	 y = "Relative Speed-dependent Voigt Absorbance")
\end{lstlisting}

\includegraphics[width=.9\linewidth]{output/hitran_PA.pdf}

Clearly, the relation between pressure and absorbance is very weak indeed.

Turning now to temperature variation yields the following values
\lstset{literate={<-}{{$\gets$}}1 {<=}{{$\leq$}}1 {>=}{{$\geq$}}1 {!=}{{$\neq$}}1 {<>}{{$\neq$}}1 {\%>\%}{{$\Leftarrow$}}1 {\%<>\%}{{$\Leftrightarrow$}}1,frame=single,language=sh,label= ,caption= ,numbers=none}
\begin{lstlisting}
cd hitran2020
rm -f ../cache/lm_calc_T.dat

cat > lm_input.stub <<EOF
SgMinR  590.0
SgMaxR  740.0
dSg     0.01
sTotMax 0.1e-28
Ptot    1.0
xCO2    420.e-6
xH2O    0
MixFull 0
MixSDV  1
Temp    290
EOF

for T in 290 280 270 260 250 240 230 220 210 200; do
    head -9 lm_input.stub > lm_input.in
    cat >> lm_input.in <<EOF
Temp    $T
EOF

./lm_calc

awk -v Temp=$T '{AbsV += $2;AbsL += $3}END{print Temp, AbsV/100, AbsL/100}' \
    ABSCO_HIT2020.dat >> ../cache/lm_calc_T.dat

done
cd ..
\end{lstlisting}

again, removing the density effect of the Temperature changes and
normalising to the base case, we get the plot
\lstset{literate={<-}{{$\gets$}}1 {<=}{{$\leq$}}1 {>=}{{$\geq$}}1 {!=}{{$\neq$}}1 {<>}{{$\neq$}}1 {\%>\%}{{$\Leftarrow$}}1 {\%<>\%}{{$\Leftrightarrow$}}1,frame=single,language=R,label= ,caption= ,numbers=none}
\begin{lstlisting}
Temp_Abs <- read.table("cache/lm_calc_T.dat",
			col.names = c("Temperature", "AbsV", "AbsL"))

Temp_Abs %>%
    mutate(Voigt = AbsV * Temperature /
	       (Temp_Abs$AbsV[1] * Temp_Abs$Temperature[1])) %>%
    ggplot() +
    geom_point(aes(Temperature, Voigt)) +
    labs(title = "Relative Absorbance in function of Temperature",
	 x = "Temperature (K)",
	 y = "Relative Speed-dependent Voigt Absorbance")
\end{lstlisting}

\includegraphics[width=.9\linewidth]{output/hitran_TA.pdf}

Once again, the influence of temperature on absorption is very
small. The last two graphs show that the relative effects of pressure
and temperature ranges encountered in the atmosphere on the absorbance
are tiny in comparison with the changes in CO$_{\text{2}}$ concentrations.

Let's check to what extent this effect plays out on the typical
atmospheric p,T profile.  Let's use the values of temperature,
pressure and water content of the earlier mentioned model atmosphere
profiles, and calculating the integrated absorbance at 420 ppm CO$_{\text{2}}$,
again \unit{150}{\centi\reciprocal\metre} around the
\unit{665}{\centi\reciprocal\metre} wave number. Note that the
pressure in the original data files is expressed in mbar, so we need
to convert to atm as expected by the HITRAN program

\lstset{literate={<-}{{$\gets$}}1 {<=}{{$\leq$}}1 {>=}{{$\geq$}}1 {!=}{{$\neq$}}1 {<>}{{$\neq$}}1 {\%>\%}{{$\Leftarrow$}}1 {\%<>\%}{{$\Leftrightarrow$}}1,frame=single,language=sh,label= ,caption= ,numbers=none}
\begin{lstlisting}
cd hitran2020
rm -f ../cache/atm_profile_results.dat

awk 'NR > 2 && $1 < 46 {print "SgMinR  590.0\nSgMaxR  740.0\ndSg     0.01\n\
TotMax  0.1e-28\nxCO2    420.e-6\nMixFull 0\nMixSDV  1\nTemp    "$3 "\n\
Ptot    "$2/1013.25 "\nxH2O    "$4}' \
../data/grl54358_atmosphere_tropical.dat > lm_input.new

for i in `awk 'BEGIN { for( i=1; i<=46; i++ ) print i }'`; do
    my_n=$(($i*10))
    my_h=$(($i-1))
    head -$my_n lm_input.new | tail -10 > lm_input.in
    ./lm_calc
    awk -v h=$my_h '{AbsV += $2;AbsL += $3}END{print h, AbsV/100, AbsL/100}'\
	ABSCO_HIT2020.dat >> ../cache/atm_profile_results.dat 
done

cd ..
\end{lstlisting}

From the atmospheric profile for a mid-latitude summer, we then obtain

\lstset{literate={<-}{{$\gets$}}1 {<=}{{$\leq$}}1 {>=}{{$\geq$}}1 {!=}{{$\neq$}}1 {<>}{{$\neq$}}1 {\%>\%}{{$\Leftarrow$}}1 {\%<>\%}{{$\Leftrightarrow$}}1,frame=single,language=R,label= ,caption= ,numbers=none}
\begin{lstlisting}
abs_profile <-
    read.table("cache/atm_profile_results.dat",
	       col.names = c("height", "AbsV", "AbsF"))

abs_profile %<>%
    left_join(atm_mid_s, by="height") %>%
    mutate(AbsV = AbsV / 150,
	   AbsF = AbsF / 150)

ggplot(abs_profile) +
    geom_line(aes(height, AbsV)) +
    labs(title = "Mid-latitude Summer Atmospheric Absorption Profile",
	 x = "Height (km)",
	 y = expression(paste('Absorption Coefficient ', (m^{-3}))))
\end{lstlisting}

\includegraphics[width=.9\linewidth]{output/abs_profile.pdf}

As the density of CO$_{\text{2}}$ is the main driver in this absorption
profile, let's see how the individual molecule absorption varies with
the atmospheric conditions, that is to say, let's show how the effects
of pressure and temperature change in the atmosphere play out on the
per molecule absorption.

\lstset{literate={<-}{{$\gets$}}1 {<=}{{$\leq$}}1 {>=}{{$\geq$}}1 {!=}{{$\neq$}}1 {<>}{{$\neq$}}1 {\%>\%}{{$\Leftarrow$}}1 {\%<>\%}{{$\Leftrightarrow$}}1,frame=single,language=R,label= ,caption= ,numbers=none}
\begin{lstlisting}
xCO2 <- 420e-6
abs_profile %<>%
    mutate(CO2 = pressure * N_A * xCO2 / (R_gas * temperature),
	   Voigt = AbsV / CO2,
	   First = AbsF / CO2)

ggplot(abs_profile) +
    geom_line(aes(height, Voigt / abs_profile$Voigt[1])) +
    labs(title = "Molecule-specific  Absorption Profile",
	 x = "Height (km)",
	 y = "Relative Absorption ")
\end{lstlisting}

\includegraphics[width=.9\linewidth]{output/abs_profile_molecule.pdf}

As expected from the slight pressure-temperature dependencies we've
seen earlier, the absorbance changes very little.

Turning now to the total absorption of this atmospheric column from
\begin{equation}
A = 1 - \prod_1^n (1 - a_i)
\end{equation}
where $a_i$ stands for the absorption in a volume element $i$. Bearing
in mind that the HITRAN calculations have been carried out for
\unit{1}{\cubic\metre} at \unit{1}{\kilo\metre} intervals in the
atmospheric column, it is a good idea to extend the calculations by
interpolating the volume element absorptions.

\lstset{literate={<-}{{$\gets$}}1 {<=}{{$\leq$}}1 {>=}{{$\geq$}}1 {!=}{{$\neq$}}1 {<>}{{$\neq$}}1 {\%>\%}{{$\Leftarrow$}}1 {\%<>\%}{{$\Leftrightarrow$}}1,frame=single,language=R,label= ,caption= ,numbers=none}
\begin{lstlisting}
band_energy <- earth_avail$energy[earth_avail$band == "0665"]

absorption <- data.frame(h = seq(0, 45, 0.001)) %>%
    mutate(AbsV = interp1(abs_profile$h, abs_profile$AbsV, h,
			  method = "spline"),
	   AbsF = interp1(abs_profile$h, abs_profile$AbsF, h,
			  method = "spline"),
	   transV = cumprod(1 - AbsV),
	   transF = cumprod(1 - AbsF)) %>%
    summarise_all(min) %>%
    mutate(Voigt = 1 - transV,
	   Line = 1 - transF) %>%
    select(Voigt, Line) %>%
    mutate(Voigt_Abs = Voigt * band_energy,
	   Line_Abs = Line * band_energy)
absorption
\end{lstlisting}

\begin{verbatim}
      Voigt      Line Voigt_Abs Line_Abs
1 0.9957845 0.9957734  38.31587 38.31545
\end{verbatim}

Now we can run a suite of different CO$_{\text{2}}$ concentrations and see how
the absorption, and absorbed energy, changes. The choice of
concentrations reflects a number of known conditions, as tabled here

\begin{center}
\begin{tabular}{lr}
\toprule
 & CO$_{\text{2}}$ (ppm)\\
\midrule
Glacial & 180\\
Interglacial & 280\\
Pliocene? & 360\\
Present Day & 420\\
\bottomrule
\end{tabular}
\end{center}

And we can extend that analysis over the different atmospheric
profiles looked at higher up. That gives us

\begin{center}
\begin{tabular}{llrrr}
\toprule
latitude & season & CO2 & Voigt & Abs\\
\midrule
mid & summer & 180 & 0.87569 & 33.631\\
mid & summer & 280 & 0.96097 & 36.906\\
mid & summer & 360 & 0.98455 & 37.811\\
mid & summer & 420 & 0.99229 & 38.109\\
mid & winter & 180 & 0.87476 & 33.596\\
mid & winter & 280 & 0.96052 & 36.889\\
mid & winter & 360 & 0.98432 & 37.803\\
mid & winter & 420 & 0.99216 & 38.104\\
subarctic & summer & 180 & 0.87381 & 33.559\\
subarctic & summer & 280 & 0.96005 & 36.871\\
subarctic & summer & 360 & 0.98408 & 37.794\\
subarctic & summer & 420 & 0.99202 & 38.099\\
subarctic & winter & 180 & 0.87157 & 33.473\\
subarctic & winter & 280 & 0.95894 & 36.828\\
subarctic & winter & 360 & 0.98351 & 37.772\\
subarctic & winter & 420 & 0.99168 & 38.086\\
tropical & summer & 180 & 0.87654 & 33.664\\
tropical & summer & 280 & 0.96138 & 36.922\\
tropical & summer & 360 & 0.98476 & 37.820\\
tropical & summer & 420 & 0.99241 & 38.114\\
tropical & winter & 180 & 0.87654 & 33.664\\
tropical & winter & 280 & 0.96138 & 36.922\\
tropical & winter & 360 & 0.98476 & 37.820\\
tropical & winter & 420 & 0.99241 & 38.114\\
\bottomrule
\end{tabular}
\end{center}

\lstset{literate={<-}{{$\gets$}}1 {<=}{{$\leq$}}1 {>=}{{$\geq$}}1 {!=}{{$\neq$}}1 {<>}{{$\neq$}}1 {\%>\%}{{$\Leftarrow$}}1 {\%<>\%}{{$\Leftrightarrow$}}1,frame=single,language=R,label= ,caption= ,numbers=none}
\begin{lstlisting}
ggplot(atm_abs_profiles) +
    geom_line(aes(CO2,Voigt, colour = latitude)) +
    facet_wrap(~season) +
    labs(title = "Total Atmospheric Absorption by latitude and season",
	 x = "CO2 (ppm)",
	 y = "Voigt-based Absorption") +
    theme(legend.position = "bottom")
\end{lstlisting}

\includegraphics[width=.9\linewidth]{output/atm_abs_profiles.pdf}

According to these calculations, and in keeping with the earlier
uncovered tiny temperature and pressure effects, latitudinal and
seasonal variation have a negligible effect on the atmospheric
absorption profile. The degree of absorption is very high, and the
variation driven by the known changes in CO$_{\text{2}}$ concentrations is
surprisingly small. If this approach and analysis is indeed correct,
that implies that the climatic system is remarkably sensitive to very
small changes in (re-)absorbed energy. The calculations here suggest
that 2-\unit{3}{\watt\per\metre\squared} are sufficient to move from
glacial-interglacial cycles to a hothouse type of climate.

In view of the unexpected and remarkable insensitivity of the
absorption to the latitude profiles and their seasonal variation, we
can just pick one of these profiles and calculate the absorptions as a
function of CO$_{\text{2}}$ concentrations with finer granularity.

\lstset{literate={<-}{{$\gets$}}1 {<=}{{$\leq$}}1 {>=}{{$\geq$}}1 {!=}{{$\neq$}}1 {<>}{{$\neq$}}1 {\%>\%}{{$\Leftarrow$}}1 {\%<>\%}{{$\Leftrightarrow$}}1,frame=single,language=sh,label= ,caption= ,numbers=none}
\begin{lstlisting}
  cd hitran2020

  for CO2 in 150 175 200 225 250 275 300 325 350 375 400 425 450 475 500; do
      awk -v cco2=$CO2 'NR > 2 && $1 < 46 {print "SgMinR  590.0\nSgMaxR  740.0\n\
dSg     0.01\n\TotMax  0.1e-28\nxCO2    "cco2".e-6\nMixFull 0\nMixSDV  1\n\
Temp    "$3 "\nPtot    "$2/1013.25 "\nxH2O    "$4}' \
	  ../data/grl54358_atmosphere_midlat_summer.dat > lm_input.new

      for i in `awk 'BEGIN { for( i=1; i<=46; i++ ) print i }'`; do
	  my_n=$(($i*10))
	  my_h=$(($i-1))
	  head -$my_n lm_input.new | tail -10 > lm_input.in
	  ./lm_calc
	  awk -v h=$my_h '{AbsV += $2;AbsL += $3}\
END{print h, AbsV/100, AbsL/100}'\
	      ABSCO_HIT2020.dat >> ../cache/atm_profile_ppm.dat 
      done
  done
  cd ..
\end{lstlisting}

Let's process the data generated by this script (I suspect there is a
more elegant way in which to carry out the interpolations on the
different columns, something with the 'apply' function, but this will
do for now)
\lstset{literate={<-}{{$\gets$}}1 {<=}{{$\leq$}}1 {>=}{{$\geq$}}1 {!=}{{$\neq$}}1 {<>}{{$\neq$}}1 {\%>\%}{{$\Leftarrow$}}1 {\%<>\%}{{$\Leftrightarrow$}}1,frame=single,language=R,label= ,caption= ,numbers=none}
\begin{lstlisting}
ppm_profiles <- read.table("cache/atm_profile_ppm.dat",
			   col.names = c("ppm", "height", "AbsV", "AbsF"))

ppm_abs_profiles <- 
    ppm_profiles %>%
    mutate(AbsV = AbsV / 150,
	   AbsF = AbsF / 150) %>%
    select(-AbsF) %>%
    spread(ppm, AbsV)

ppm_abs <- data.frame(h = seq(0, 45, 0.001)) %>%
    mutate(`150` = 1 - interp1(ppm_abs_profiles$height, ppm_abs_profiles$`150`,
			       h, method = "spline"),
	   `175` = 1 - interp1(ppm_abs_profiles$height, ppm_abs_profiles$`175`,
			       h, method = "spline"),
	   `200` = 1 - interp1(ppm_abs_profiles$height, ppm_abs_profiles$`200`,
			       h, method = "spline"),
	   `225` = 1 - interp1(ppm_abs_profiles$height, ppm_abs_profiles$`225`,
			       h, method = "spline"),
	   `250` = 1 - interp1(ppm_abs_profiles$height, ppm_abs_profiles$`250`,
			       h, method = "spline"),
	   `275` = 1 - interp1(ppm_abs_profiles$height, ppm_abs_profiles$`275`,
			       h, method = "spline"),
	   `300` = 1 - interp1(ppm_abs_profiles$height, ppm_abs_profiles$`300`,
			       h, method = "spline"),
	   `325` = 1 - interp1(ppm_abs_profiles$height, ppm_abs_profiles$`325`,
			       h, method = "spline"),
	   `350` = 1 - interp1(ppm_abs_profiles$height, ppm_abs_profiles$`350`,
			       h, method = "spline"),
	   `375` = 1 - interp1(ppm_abs_profiles$height, ppm_abs_profiles$`375`,
			       h, method = "spline"),
	   `400` = 1 - interp1(ppm_abs_profiles$height, ppm_abs_profiles$`400`,
			       h, method = "spline"),
	   `425` = 1 - interp1(ppm_abs_profiles$height, ppm_abs_profiles$`425`,
			       h, method = "spline"),
	   `450` = 1 - interp1(ppm_abs_profiles$height, ppm_abs_profiles$`450`,
			       h, method = "spline"),
	   `475` = 1 - interp1(ppm_abs_profiles$height, ppm_abs_profiles$`475`,
			       h, method = "spline"),
	   `500` = 1 - interp1(ppm_abs_profiles$height, ppm_abs_profiles$`500`,
			       h, method = "spline")) %>%
    select(-h) %>%
    mutate_each(funs(cumprod)) %>%
    summarise_all(min) %>%
    gather(ppm, AbsV) %>%
    mutate(ppm = as.numeric(ppm),
	   AbsV = 1 - AbsV,
	   Voigt_Abs = AbsV * band_energy)
\end{lstlisting}

And so we get the following relationship between CO$_{\text{2}}$ concentration
and absorbed energy by the entire atmospheric column
\lstset{literate={<-}{{$\gets$}}1 {<=}{{$\leq$}}1 {>=}{{$\geq$}}1 {!=}{{$\neq$}}1 {<>}{{$\neq$}}1 {\%>\%}{{$\Leftarrow$}}1 {\%<>\%}{{$\Leftrightarrow$}}1,frame=single,language=R,label= ,caption= ,numbers=none}
\begin{lstlisting}
ppm_abs %>%
    ggplot() +
    geom_line(aes(ppm, Voigt_Abs)) +
    labs(title = "Concentration Dependence of Atmospheric Voigt Absorption",
	 x = "CO2 (ppm)",
	 y = "Absorbed Energy (W/m2)")
\end{lstlisting}

\includegraphics[width=.9\linewidth]{output/abs_voigt_ppm.pdf}

This is clearly a relation of the type
\begin{equation}
y = 1 - e^{-a x}
\end{equation}
which we can estimate from the calculated numbers
\lstset{literate={<-}{{$\gets$}}1 {<=}{{$\leq$}}1 {>=}{{$\geq$}}1 {!=}{{$\neq$}}1 {<>}{{$\neq$}}1 {\%>\%}{{$\Leftarrow$}}1 {\%<>\%}{{$\Leftrightarrow$}}1,frame=single,language=R,label= ,caption= ,numbers=none}
\begin{lstlisting}
summary(lm(log(1-AbsV) ~ ppm, data = ppm_abs))
\end{lstlisting}

\begin{verbatim}
Call:
lm(formula = log(1 - AbsV) ~ ppm, data = ppm_abs)

Residuals:
       Min         1Q     Median         3Q        Max 
-9.982e-05 -3.753e-05  1.495e-05  4.731e-05  6.266e-05 

Coefficients:
              Estimate Std. Error   t value Pr(>|t|)    
(Intercept)  4.976e-04  4.713e-05     10.56 9.54e-08 ***
ppm         -1.302e-02  1.376e-07 -94630.67  < 2e-16 ***
---
Signif. codes:  0 ‘***’ 0.001 ‘**’ 0.01 ‘*’ 0.05 ‘.’ 0.1 ‘ ’ 1

Residual standard error: 5.757e-05 on 13 degrees of freedom
Multiple R-squared:      1,	Adjusted R-squared:      1 
F-statistic: 8.955e+09 on 1 and 13 DF,  p-value: < 2.2e-16
\end{verbatim}
And so we obtain as (empirical) equation
\begin{equation}
\mathrm{AbsV} = 1 - e^{-0.0130 \mathrm{ppm}}
\label{eqn:rel_absorption}
\end{equation}
from which we get the actual amount of absorbed energy by just
multiplying the CO$_{\text{2}}$ dependent AbsV value with the amount of energy
available in the spectral band, i.e.,
\unit{590-740}{\centi\reciprocal\metre}.

\subsection{Implications}
\label{sec-3-5}
These numbers and equations allow the estimation of (black body)
equilibrium temperatures, given CO$_{\text{2}}$ concentrations.

For most of the Holocene, we had 280 ppm of CO$_{\text{2}}$ with an average
temperature of \unit{287}{\kelvin}. At this temperature, a total of
\unit{38.478}{\watt\per\metre\squared} is available for absorption. The
relative absorption according to equation \ref{eqn:rel_absorption}
amounts to 0.97375 for 280 ppm (the present 420 ppm yields 0.99575,
that is to say it absorbs essentially all the energy in the relevant band).

Therefore, the stability of the Holocene climate corresponds to an
average absorption of \unit{37.468}{\watt\per\metre\squared}. In view
of the 10000 year stability, we may hypothesise that this amount of
absorbed energy is a thermal equilibrium value.

The present 420 ppm concentration of CO$_{\text{2}}$ corresponds to
\unit{38.314}{\watt\per\metre\squared}. As this is additional energy,
(\unit{0.846}{\watt\per\metre\squared}), it heats up the earth until a
new equilibrium temperature is reached. As shown earlier, a hotter
(black) body emits less energy in the CO$_{\text{2}}$ absorption band:
presumably, the heating will stop when the earth is hot enough for
there to be no longer an excess energy above the equilibrium value. We
can therefore use the calculations behind table \ref{energy665}, and
equation \ref{eqn:rel_absorption} to predict the equilibrium black
body temperature for a given CO$_{\text{2}}$ concentration.

\lstset{literate={<-}{{$\gets$}}1 {<=}{{$\leq$}}1 {>=}{{$\geq$}}1 {!=}{{$\neq$}}1 {<>}{{$\neq$}}1 {\%>\%}{{$\Leftarrow$}}1 {\%<>\%}{{$\Leftrightarrow$}}1,frame=single,language=R,label= ,caption= ,numbers=none}
\begin{lstlisting}
E_equilibrium <- 37.468

T_ppm_prediction <- data.frame(T = seq(275, 299, 1)) %>%
    mutate(Energy = interp1(energy665$T, energy665$Energy,
			    T, method = "spline"),
	   ppm = -log(1 - E_equilibrium/Energy) / 0.013)

ggplot(T_ppm_prediction) +
    geom_line(aes(ppm, T)) +
    labs(title = "Predicted Black Body T for given [CO2]",
	 x = "CO2 (ppm)",
	 y = "Temperature (K)")
\end{lstlisting}

\includegraphics[width=.9\linewidth]{output/T_ppm_prediction.pdf}

This plot proposes that the Earth's average temperature would reach a
maximum of \unit{300}{\kelvin}. Once the atmospheric CO$_{\text{2}}$
concentration reaches the 450 ppm mark, all the available band energy
will be absorbed, and higher concentrations will make no additional
contribution.

If this analysis is substantially correct, the implications for action
to stop climate change are stark. Even if net zero emission of
greenhouse gases were achieved today, temperatures will continue to
rise to reach \unit{298}{\kelvin}. To maintain the current average
temperature of \unit{288}{\kelvin} we not only have to stop emissions,
we will have to reduce the CO$_{\text{2}}$ concentration in the atmosphere down
to 285 ppm. That translates to the sequestration of
\unit{1.09}{\petad\kilogram}.

\section{References}
\label{sec-4}
The org-mode source and associated files reside at \url{https://github.com/stefan-revets/climate}

\bibliographystyle{plainnat}
\bibliography{climate}
% Emacs 24.5.1 (Org mode 8.2.10)
\end{document}